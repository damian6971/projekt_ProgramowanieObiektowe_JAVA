\chapter{Harmonogram i Zarządzanie Projektem}
\label{cha:harmonogramizarzadzanie}
%---------------------------------------------------------------------------

\section{Harmonogram realizacji – Diagram Gantta}
\label{sec:diagramGanta}
Proces tworzenia aplikacji podzielono na cztery etapy, z jasno określonym celem i udziałem czasowym. Harmonogram zapewniał systematyczną realizację prac.

\subsection{Faza 1: Analiza i projektowanie (20\%) }
\label{sec:wykorzystaneTechnologie}
Przeprowadzono analizę wymagań, określono funkcje systemu, zaprojektowano strukturę bazy danych oraz architekturę aplikacji.



\subsection{Faza 2: Implementacja warstwy danych i logiki (25\%)}
\label{sec:wykorzystaneTechnologie}
Stworzono warstwę DAO oraz klasy modelowe dla sprzętu, użytkowników i rezerwacji, zapewniając komunikację z bazą danych.

\subsection{Faza 3: Interfejs graficzny (45\%)}
\label{sec:wykorzystaneTechnologie}
Zbudowano GUI w Java Swing — panele logowania, rejestracji, rezerwacji, zarządzania sprzętem i użytkownikami, z walidacją danych.



\subsection{Faza 4: Testowanie i dokumentacja (10\%)}
\label{sec:wykorzystaneTechnologie}
Przeprowadzono testy funkcjonalne (rejestracja, logowanie, rezerwacje, paragony), a także opracowano dokumentację projektu.

% ------------------------
\section{Napotkane wyzwania i rozwiązania}
\label{sec:wyzwaniairozwiazania}
W trakcie robienia projektu pojawił się problem z wyborem prostego i wygodnego sposobu wybierania daty przy rezerwacji. Na początku próbowano zwykłych przycisków i list rozwijanych z biblioteki Swing, ale były niewygodne. Ostatecznie użyto biblioteki LGoodDatePicker, która ma fajny, czytelny kalendarz. Dzięki temu obsługa rezerwacji stała się dużo wygodniejsza.
% ------------------------
\section{System kontroli wersji i repozytorium}
\label{sec:Systemkontroliwersji}
Praca nad kodem odbywała się głównie w środowisku programistycznym IntelliJ IDEA, które zostało skonfigurowane do pracy z Git. Repozytorium projektu było publicznie dostępne pod adresem: \\
\url{https://github.com/damian6971/projekt_ProgramowanieObiektowe_JAVA}