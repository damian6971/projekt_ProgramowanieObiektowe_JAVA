\chapter{Wnioski końcowe}
\label{cha:wnioskiKoncowe}

\section{Podsumowanie projektu}
\label{sec:podsumowanieProjektu}
Celem projektu było stworzenie w pełni funkcjonalnej aplikacji działającej jako wypożyczalnia sprzętu budowlanego. System został zaprojektowany i zrealizowany zgodnie z założeniami – użytkownicy mają możliwość rejestracji, rezerwacji sprzętu oraz zarządzania swoimi danymi, natomiast administrator posiada dostęp do rozszerzonych funkcji związanych z obsługą rezerwacji, wypożyczeń, listy sprzętu oraz kont użytkowników. Projekt został zrealizowany zgodnie z harmonogramem, a funkcjonalność systemu została potwierdzona w procesie testowania. W trakcie realizacji projektu zauważono kilka rzeczy, które w przyszłości będą mogły zostać uelpszone – zarówno pod względem działania aplikacji, jak i wygody użytkownika.


\section{Możliwość rozbudowy}
\label{sec:mozliwoscRozbudowy}
W kolejnych etapach rozwoju systemu można rozważyć wdrożenie następujących funkcjonalności:
\begin{enumerate}
    \item Logowanie za pomocą Google/Facebook – uproszczenie procesu rejestracji i logowania,
    \item System ocen i komentarzy użytkowników – umożliwienie wystawiania opinii o wypożyczonym sprzęcie,
    \item Umożliwienie użytkownikom dokonywania płatności online,
    \item Możliwość resetowania hasła dla użytkowników - Dodanie funkcji przywracania dostępu do konta bez konieczności kontaktu z administratorem,
    \item Automatyczna akceptacja rezerwacji – system mógłby samodzielnie zatwierdzać rezerwacje, jeśli spełnione są określone warunki, np. sprzęt jest dostępny, użytkownik nie ma zaległości, a termin nie koliduje z innymi wypożyczeniami,
    \item Aplikacja mobilna lub wersja webowa - Umożliwienie użytkownikom składania rezerwacji przez telefon lub przeglądarkę, niezależnie od systemu operacyjnego.
\end{enumerate}